\documentclass{article}

\usepackage[utf8]{inputenc}
\usepackage{amsmath}
\usepackage{amsfonts}
\usepackage{amssymb}
\usepackage{graphicx}
\usepackage[table,xcdraw]{xcolor}
\usepackage[bookmarks,hypertexnames=false,debug,linktocpage=true,hidelinks]{hyperref}
\usepackage{placeins}

\hypersetup{
    colorlinks,
    linktoc=all,
    linkcolor={blue},
    citecolor={blue},
    urlcolor={blue}
}

\usepackage{tikz}
\newcommand*\circled[1]{\tikz[baseline=(char.base)]{
            \node[shape=circle,draw,inner sep=2pt] (char) {#1};}}


\graphicspath{ {./images/} }

\renewcommand{\contentsname}{Indice}

\makeatletter
\newcommand*{\rom}[1]{\expandafter\@slowromancap\romannumeral #1@}
\makeatother

\usepackage[a4paper,top=2cm,bottom=2cm,left=2cm,right=2cm]{geometry}

\setcounter{section}{-1}

\title{\textbf{\Huge Specifica dei Requisiti}}
\author{Matteo Girardi, Vasile Donmovil, Antonio Scendrate Gattico \\ Gruppo T45 - Deliverable D2}
\date{2022}

\begin{document}

\maketitle

\clearpage
\tableofcontents
\clearpage

\section{Scopo del documento}
\begin{description}
    \item[] Il presente documento illustra la specifica dei requisiti di sistema del progetto, utilizzando anche rappresentazioni grafiche, come diagrammi UML e tabelle strutturate; i requisiti verranno descritti usando sia linguaggio naturale sia in linguaggi più formali e strutturati.
    \item[] Grazie a diagrammi di contesto e dei componenti verrà presentato il design del sistema.
\end{description}
\clearpage

\section{Requisiti Funzionali}


\begin{description}
	\item[] Si individuano 3 tipi di utenti che interagiranno con la webapp:
		\begin{itemize}
		\item  \circled{Un} Utente Non Registrato
		\item  \circled{Re} Utente Registrato
		\item  \circled{Ad} Utente Amministratore
		\end{itemize}
\renewcommand\thesubsubsection{RF\arabic{subsubsection}}
\subsubsection{Visualizzazione Date} \label{rf_1}
\begin{description}
    \item Questo requisito è condivido da tutti i tipi di utente.
\end{description}

\subsubsection{Prenotazione Sala} \label{rf_2}
\begin{description}
    \item Questo requisito appartiene a: \circled{Re}
\end{description}


\subsubsection{Gestione prenotazioni} \label{rf_3}
\begin{description}
    \item Questo requisito appartiene a: \circled{Re} \circled{Ad}
\end{description}

\subsubsection{Annullare una Prenotazione} \label{rf_4}
\begin{description}
    \item Questo requisito appartiene a: \circled{Re} \circled{Ad}
\end{description}

\subsubsection{Login} \label{rf_5}
\begin{description}
    \item Questo requisito appartiene a: \circled{Re} \circled{Ad}
\end{description}

\subsubsection{Logout} \label{rf_6}
\begin{description}
    \item Questo requisito appartiene a: \circled{Re} \circled{Ad}
\end{description}

\subsubsection{Creazione Nuovo Account} \label{rf_7}
\begin{description}
    \item Questo requisito appartiene a: \circled{Ad}
\end{description}

\subsubsection{Gestione Account} \label{rf_8}
\begin{description}
    \item Questo requisito appartiene a: \circled{Re} \circled{Ad}
\end{description}

\subsubsection{Rimozione Account} \label{rf_9}
\begin{description}
    \item[] I donatori registrati potranno scegliere di cancellare il proprio account.
    \item Questo requisito appartiene a: \circled{Re} \circled{Ad}
\end{description}

\subsubsection{Modifica Stato Sala} \label{rf_10}
\begin{description}
    \item Questo requisito appartiene a: \circled{Ad}
\end{description}

\subsubsection{Notifica Prenotazione} \label{rf_11}
\begin{description}
    \item Questo requisito appartiene a: \circled{Re}
\end{description}



\end{description}



\clearpage

\end{document}