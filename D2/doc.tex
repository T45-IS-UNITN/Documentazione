\documentclass{article}

\usepackage[utf8]{inputenc}
\usepackage{amsmath}
\usepackage{amsfonts}
\usepackage{amssymb}
\usepackage{graphicx}
\usepackage[table,xcdraw]{xcolor}
\usepackage[bookmarks,hypertexnames=false,debug,linktocpage=true,hidelinks]{hyperref}
\usepackage{placeins}

\hypersetup{
    colorlinks,
    linktoc=all,
    linkcolor={blue},
    citecolor={blue},
    urlcolor={blue}
}

\usepackage{tikz}
\newcommand*\circled[1]{\tikz[baseline=(char.base)]{
            \node[shape=circle,draw,inner sep=2pt] (char) {#1};}}


\graphicspath{ {./media/} }

\renewcommand{\contentsname}{Indice}

\makeatletter
\newcommand*{\rom}[1]{\expandafter\@slowromancap\romannumeral#1@}
\makeatother

\usepackage[a4paper,top=2cm,bottom=2cm,left=2cm,right=2cm]{geometry}

\setcounter{section}{-1}

\title{\textbf{\Huge Specifica dei Requisiti}}
\author{Matteo Girardi, Vasile Donmovil, Antonio Scendrate Gattico \\ Gruppo T45 - Deliverable D2}
\date{2022}

\begin{document}

\maketitle

\clearpage
\tableofcontents
\clearpage

% #### Scopo del documento ####
\section{Scopo del documento}
\begin{description}
    \item[] Il presente documento illustra la specifica dei requisiti di sistema del progetto, utilizzando anche rappresentazioni grafiche, come diagrammi UML e tabelle strutturate; i requisiti verranno descritti usando sia linguaggio naturale sia in linguaggi più formali e strutturati.
    \item[] Grazie a diagrammi di contesto e dei componenti verrà presentato il design del sistema.
\end{description}
\clearpage

% #### Req. funz ####
\section{Requisiti Funzionali}
\begin{description}
	\item[] Si individuano 4 attori che interagiranno con la webapp:
		\begin{itemize}
		\item  Utente Non Registrato
		\item  Utente Registrato
		\item  Utente Amministratore
		\item  Servizio di Notifica
		\end{itemize}
\end{description}


% # Utente non registrato #
\subsection{Utente Non Registrato}
\renewcommand\thesubsubsection{RF \arabic{subsubsection}}

\subsubsection{Visualizzazione Date}\label{rf_1}
\begin{description}
    
    \begin{figure}[htp]
	\centering
	\includegraphics[]{rf1.png}
	\end{figure}
     
    \item L'utente può visualizzare un calendario con evidenziati i giorni in cui sono disponibili gli appuntamenti. 
\end{description}


% # Utente registrato #
\subsection{Utente Registrato}

\subsubsection{Visualizzazione Date}\label{rf_1}
\begin{description}
    
    \begin{figure}[htp]
	\centering
	\includegraphics[]{rf1.png}
	\end{figure}
    
    \item L'utente può visualizzare un calendario con evidenziati i giorni in cui sono disponibili gli appuntamenti. 
\end{description}

\subsubsection{Prenotazione Sala}\label{rf_2}
\begin{description}

	 \begin{figure}[htp]
	\centering
	\includegraphics[]{rf2.png}
	\end{figure}
	
    \item Un utente è in grado di prenotare un appuntamento, consultando il calendario dove sono evidenziate le date disponibili. L'utente può concludere con successo una prenotazione solamente dopo aver effettuato il Login al sistema.   	
\end{description}

\clearpage

\subsubsection{Gestione prenotazioni}\label{rf_3}
\begin{description}

	\begin{figure}[htp]
	\centering
	\includegraphics[width=\textwidth]{rf3.png}
	\end{figure}

    \item L'utente può consultare le proprie prenotazioni, precedentemente prenotate, così da poter capire quando e dove avverranno.
\end{description}

\subsubsection{Annullare una Prenotazione}\label{rf_4}
\begin{description}

	\begin{figure}[htp]
	\centering
	\includegraphics[]{rf4.png}
	\end{figure}

    \item L'utente può annullare le proprie prenotazioni precedentemente effettuate.
\end{description}


\subsubsection{Login}\label{rf_5}
\begin{description}

	\begin{figure}[htp]
	\centering
	\includegraphics[]{rf5.png}
	\end{figure}

    \item L'utente che possiede credenziali idonee può utilizzarle per autenticarsi alla piattaforma, così da poter interagire con essa da utente registrato.
\end{description}

\clearpage

\subsubsection{Logout}\label{rf_6}
\begin{description}

	\begin{figure}[htp]
	\centering
	\includegraphics[]{rf6.png}
	\end{figure}

    \item Una volta che l'utente ha eseguito l'accesso alla piattaforma è anche in grado di terminare la sessione, di fatto tornando un utente non registrato agli occhi di quest'ultima. L'utente deve aver effettuato il login precedentemente. 
\end{description}


\renewcommand\thesubsubsection{RF 8}
\subsubsection{Gestione Account}\label{rf_8}
\begin{description}

	\begin{figure}[htp]
	\centering
	\includegraphics[]{rf8.png}
	\end{figure}	
	
    \item L'utente che ha effettuato il login al sistema è anche in grado di consultare i propri dati interni al sistema, potendo anche aggiornare tali dati.
\end{description}

\renewcommand\thesubsubsection{RF 9}
\subsubsection{Rimozione Account}\label{rf_9}
\begin{description}

	\begin{figure}[htp]
	\centering
	\includegraphics[]{rf9.png}
	\end{figure}

    \item L'utente che ha effettuato il login al sistema è anche in grado di cancellare il proprio account, così da terminare la sua capacità di usufruire della piattaforma.
\end{description}

\clearpage

% # ADMIN #
\subsection{Utente Amministratore}

\renewcommand\thesubsubsection{RF \arabic{subsubsection}}
\subsubsection{Visualizzazione Date}\label{rf_1}
\begin{description}
    
    \begin{figure}[htp]
	\centering
	\includegraphics[]{rf1.png}
	\end{figure}
    
    \item L'utente può visualizzare un calendario con evidenziati i giorni in cui sono disponibili gli appuntamenti. 
\end{description}

\renewcommand\thesubsubsection{RF 3}
\subsubsection{Gestione prenotazioni}\label{rf_3}
\begin{description}

	\begin{figure}[htp]
	\centering
	\includegraphics[width=\textwidth]{rf3.png}
	\end{figure}

    \item L'utente Amministratore può consultare le prenotazioni effettuate dagli utenti registrati alla piattaforma.
\end{description}

\renewcommand\thesubsubsection{RF 4}
\subsubsection{Annullare una Prenotazione}\label{rf_4}
\begin{description}

	\begin{figure}[htp]
	\centering
	\includegraphics[]{rf4.png}
	\end{figure}

    \item L'utente Amministratore può annullare le prenotazioni precedentemente effettuate da parte di altri utenti registrati.
\end{description}

\clearpage

\renewcommand\thesubsubsection{RF 5}
\subsubsection{Login}\label{rf_5}
\begin{description}

	\begin{figure}[htp]
	\centering
	\includegraphics[]{rf5.png}
	\end{figure}

    \item L'utente Amministratore che possiede credenziali idonee può utilizzarle per autenticarsi alla piattaforma, così da poter interagire con essa da utente registrato.
\end{description}

\renewcommand\thesubsubsection{RF 6}
\subsubsection{Logout}\label{rf_6}
\begin{description}

	\begin{figure}[htp]
	\centering
	\includegraphics[]{rf6.png}
	\end{figure}

    \item Una volta che l'utente Amministratore ha eseguito l'accesso alla piattaforma è anche in grado di terminare la sessione, di fatto tornando un utente non registrato agli occhi di quest'ultima. L'utente deve aver effettuato il login precedentemente. 
\end{description}

\renewcommand\thesubsubsection{RF 7}
\subsubsection{Creazione Nuovo Account}\label{rf_7}
\begin{description}

	\begin{figure}[htp]
	\centering
	\includegraphics[]{rf7.png}
	\end{figure}

    \item L'utente amministratore può creare manualmente un nuovo account utente, somministrando alla Pubblica Amministrazione i dati anagrafici forniti dal donatore (ossia un futuro utilizzatore della piattaforma). Ogni donatore è registrato presso un ente dedicato interno alla PA. Una volta che l'ente accetta la registrazione del donatore al sistema può interagire con esso, effettuando il login.
\end{description}

\clearpage

\renewcommand\thesubsubsection{RF 8}
\subsubsection{Gestione Account}\label{rf_8}
\begin{description}

	\begin{figure}[htp]
	\centering
	\includegraphics[]{rf8.png}
	\end{figure}	
	
    \item L'utente che ha effettuato il login al sistema è anche in grado di consultare i propri dati interni al sistema, potendo anche aggiornare tali dati. L'utente Amministratore è in grado di consultare informazioni appartenenti agli utenti registrati.
\end{description}

\renewcommand\thesubsubsection{RF 9}
\subsubsection{Rimozione Account}\label{rf_9}
\begin{description}

	\begin{figure}[htp]
	\centering
	\includegraphics[]{rf9.png}
	\end{figure}

    \item L'utente che ha effettuato il login al sistema è anche in grado di cancellare il proprio account, così da terminare la sua capacità di usufruire della piattaforma. L'utente Amministratore è in grado di cancellare l'account di un utente regsitrato.
\end{description}

\renewcommand\thesubsubsection{RF 10}
\subsubsection{Modifica Stato Sala}\label{rf_10}
\begin{description}

	\begin{figure}[htp]
	\centering
	\includegraphics[]{rf10.png}
	\end{figure}

    \item L’utente amministratore può modificare lo stato di una specifica sala.
\end{description}

\clearpage

\subsection{Servizio di Notifica SMS/Email}
% ### notifica ###
\renewcommand\thesubsubsection{RF 11}
\subsubsection{Notifica Prenotazione}\label{rf_11}
\begin{description}

	\begin{figure}[htp]
	\centering
	\includegraphics[]{rf11.png}
	\end{figure}
	
	\item Quando la prenotazione è imminente, il servizio di notifica tramite SMS/Email invia un messaggio all'utente riguardo l'evento.
\end{description}
\clearpage


% #### Req. non. funz ####
\section{Requisiti Non Funzionali}
\begin{description}
    \item[] Nel presente capitolo vengono riportati i requisiti non funzionali (RNF) del sistema utilizzando tabelle strutturate e specificando misure facilmente verificabili. 
\end{description}

\renewcommand\thesubsubsection{RNF \arabic{subsubsection}}

\subsubsection{Privacy}\label{rnf_1}
\begin{description}
    \item Privacy
\end{description}

\subsubsection{Sicurezza}\label{rnf_2}
\begin{description}
    \item Privacy
\end{description}

\subsubsection{Scalabilità}\label{rnf_3}
\begin{description}
    \item Privacy
\end{description}

\subsubsection{Affidabilità}\label{rnf_4}
\begin{description}
    \item Privacy
\end{description}

\subsubsection{Resilienza}\label{rnf_5}
\begin{description}
    \item Privacy
\end{description}

\subsubsection{Accessibilità}\label{rnf_6}
\begin{description}
    \item Privacy
\end{description}

\subsubsection{Monitoraggio}\label{rnf_7}
\begin{description}
    \item Privacy
\end{description}

\subsubsection{Multilingua}\label{rnf_8}
\begin{description}
    \item Privacy
\end{description}

\clearpage

% #### Anal. Contesto ####
\section{Analisi del Contesto}
\begin{description}
    \item[] Nel presente capitolo viene discusso il contesto di funzionamento del sistema, fornendo una descrizione testuale e una rappresentazione grafica basata su Context Diagram.

    Nella seguente parte della sezione vengono presentati gli attori e i sistemi esterni con cui BloodStream si interfaccerà. 
\end{description}

\subsection{Utenti e Sistemi esterni}
\subsection{Diagramma di contesto}
\clearpage

% #### Anal. Compoonent. ####
\section{Analisi dei Componenti}
\subsection{Definizione dei componenti}

\subsection{Diagramma dei componenti}

\end{document}