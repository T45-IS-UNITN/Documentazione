\documentclass{article}

\usepackage[utf8]{inputenc}
\usepackage{amsmath}
\usepackage{amsfonts}
\usepackage{amssymb}
\usepackage{graphicx}
\usepackage[table,xcdraw]{xcolor}
\usepackage[bookmarks,hypertexnames=false,debug,linktocpage=true,hidelinks]{hyperref}
\usepackage{placeins}

\hypersetup{
    colorlinks,
    linktoc=all,
    linkcolor={blue},
    citecolor={blue},
    urlcolor={blue}
}

\usepackage{tikz}
\newcommand*\circled[1]{\tikz[baseline=(char.base)]{
            \node[shape=circle,draw,inner sep=2pt] (char) {#1};}}


\graphicspath{ {./images/} }

\renewcommand{\contentsname}{Indice}

\makeatletter
\newcommand*{\rom}[1]{\expandafter\@slowromancap\romannumeral#1@}
\makeatother

\usepackage[a4paper,top=2cm,bottom=2cm,left=2cm,right=2cm]{geometry}

\setcounter{section}{-1}

\title{\textbf{\Huge Specifica dei Requisiti}}
\author{Matteo Girardi, Vasile Donmovil, Antonio Scendrate Gattico \\ Gruppo T45 - Deliverable D2}
\date{2022}

\begin{document}

\maketitle

\clearpage
\tableofcontents
\clearpage

% #### Scopo del documento ####
\section{Scopo del documento}
\begin{description}
    \item[] Il presente documento illustra la specifica dei requisiti di sistema del progetto, utilizzando anche rappresentazioni grafiche, come diagrammi UML e tabelle strutturate; i requisiti verranno descritti usando sia linguaggio naturale sia in linguaggi più formali e strutturati.
    \item[] Grazie a diagrammi di contesto e dei componenti verrà presentato il design del sistema.
\end{description}
\clearpage

% #### Req. funz ####
\section{Requisiti Funzionali}
\begin{description}
	\item[] Si individuano 3 tipi di utenti che interagiranno con la webapp:
		\begin{itemize}
		\item  \circled{Un} Utente Non Registrato
		\item  \circled{Re} Utente Registrato
		\item  \circled{Ad} Utente Amministratore
		\end{itemize}
\end{description}

\renewcommand\thesubsection{RF\arabic{subsection}}

\subsection{Visualizzazione Date}\label{rf_1}
\begin{description}
    \item Questo requisito è condivido da tutti i tipi di utente.
\end{description}

\subsection{Prenotazione Sala}\label{rf_2}
\begin{description}
    \item Questo requisito appartiene a: \circled{Re}
\end{description}

\subsection{Gestione prenotazioni}\label{rf_3}
\begin{description}
    \item Questo requisito appartiene a: \circled{Re} \circled{Ad}
\end{description}

\subsection{Annullare una Prenotazione}\label{rf_4}
\begin{description}
    \item Questo requisito appartiene a: \circled{Re} \circled{Ad}
\end{description}

\subsection{Login}\label{rf_5}
\begin{description}
    \item Questo requisito appartiene a: \circled{Re} \circled{Ad}
\end{description}

\subsection{Logout}\label{rf_6}
\begin{description}
    \item Questo requisito appartiene a: \circled{Re} \circled{Ad}
\end{description}

\subsection{Creazione Nuovo Account}\label{rf_7}
\begin{description}
    \item Questo requisito appartiene a: \circled{Ad}
\end{description}

\subsection{Gestione Account}\label{rf_8}
\begin{description}
    \item Questo requisito appartiene a: \circled{Re} \circled{Ad}
\end{description}

\subsection{Rimozione Account}\label{rf_9}
\begin{description}
    \item[] I donatori registrati potranno scegliere di cancellare il proprio account.
    \item Questo requisito appartiene a: \circled{Re} \circled{Ad}
\end{description}

\subsection{Modifica Stato Sala}\label{rf_10}
\begin{description}
    \item Questo requisito appartiene a: \circled{Ad}
\end{description}

\subsection{Notifica Prenotazione}\label{rf_11}
\begin{description}
    \item Questo requisito appartiene a: \circled{Re}
\end{description}

\clearpage

% #### Req. non. funz ####
\section{Requisiti Non Funzionali}
\begin{description}
    \item[] Nel presente capitolo vengono riportati i requisiti non funzionali (RNF) del sistema utilizzando tabelle strutturate e specificando misure facilmente verificabili. 
\end{description}

\renewcommand\thesubsection{RNF\arabic{subsection}}

\subsection{Privacy}\label{rnf_1}
\begin{description}
    \item Privacy
\end{description}

\subsection{Sicurezza}\label{rnf_2}
\begin{description}
    \item Privacy
\end{description}

\subsection{Scalabilità}\label{rnf_3}
\begin{description}
    \item Privacy
\end{description}

\subsection{Affidabilità}\label{rnf_4}
\begin{description}
    \item Privacy
\end{description}

\subsection{Resilienza}\label{rnf_5}
\begin{description}
    \item Privacy
\end{description}

\subsection{Accessibilità}\label{rnf_6}
\begin{description}
    \item Privacy
\end{description}

\subsection{Monitoraggio}\label{rnf_7}
\begin{description}
    \item Privacy
\end{description}

\subsection{Multilingua}\label{rnf_8}
\begin{description}
    \item Privacy
\end{description}

\clearpage


% #### Anal. Contesto ####
\section{Analisi del Contesto}
\begin{description}
    \item[] Nel presente capitolo viene discusso il contesto di funzionamento del sistema, fornendo una descrizione testuale e una rappresentazione grafica basata su Context Diagram.

    Nella seguente parte della sezione vengono presentati gli attori e i sistemi esterni con cui BloodStream si interfaccerà. 
\end{description}

\subsection{Utenti e Sistemi esterni}
\subsection{Diagramma di contesto}
\clearpage

% #### Anal. Compoonent. ####
\section{Analisi dei Componenti}
\subsection{Definizione dei componenti}

\subsection{Diagramma dei componenti}

\end{document}